\section{Modeling and Managing ATSs - The CATSM Problem} 

% ==================///==================///==================///
\begin{frame}{How to tackle them?}
	By observing what we have. \\
	
	\vspace{0.5cm}
	\begin{itemize}
		\item Road Network
		\item Vehicles
		\item Requests
	\end{itemize}
	\vspace{0.5cm}
	$\rightarrow$ Creating a vehicle-centric model of the ATS to solve the challenges.
\end{frame}

% ==================///==================///==================///
\begin{frame}{Modeling the Road Network}
	Using a uni-directed graph $\mathcal{G} = \langle \mathcal{V}, \mathcal{E} \rangle$, where $\mathcal{V}$ represents the set of vertices (locations) and $\mathcal{E} \subseteq \mathcal{V} \times \mathcal{V}$ represents the edges (roads).
	\vspace{0.5cm}
	\begin{itemize}
		\item Each edge is associated with multiple metrics (e.g., distance $d: \mathcal{E} \rightarrow \mathbb{R}_{\geq 0}$).
		\begin{itemize}
			\item[ ] $\Rightarrow$ Artificial limit on number of vehicles at each time
		\end{itemize}
		\item Nodes can be of two types: charging ($\mathcal{V}_c$) and normal ($\mathcal{V}_n$) nodes.
		\item Charging nodes using ad-hoc charging profile models.
	\end{itemize}
\end{frame}


% ==================///==================///==================///

%\begin{frame}{Charging Profiles}
%	 	\vspace{0.1cm}
%	AV batteries modeled as tuple  $\mathcal{T}_a =\langle Q_a, I^b_a, R^-_a, R^+_a,\theta_a\rangle$. \\
%	Using the CC-CV (Constant current - Constant Voltage) scheme. 
%	\vspace{0.1cm}
%	\input{img/half_battery_profiles}
%\end{frame}

% ==================///==================///==================///

\begin{frame}{Modeling AVs}
	Each vehicle $a \in\mathcal{A}$ having: %as a tuple $\langle \underline{s_a},\bar{t_a}, B_a(t),\mathcal{R}_a, \mathcal{T}_a, P_a, G_a, C_a, F_a  \rangle$.
	\begin{itemize}
		\item Starting and terminating nodes $\underline{s_a}$ and $\bar{t_a}$ 
		\item  State of Charge $B_a \in \mathbb{R}_{>0}$ at time $t$
		\item  Goods and people capacity $G_a \in \mathbb{R}_{\ge0}$ and $P_a \in \mathbb{R}_{\ge0}$ 
		\item  Operational cost $C_a \in \mathbb{R}_{>0}$ 
		\item  Pollution factor $F_a \in \mathbb{R}_{>0}$
		%\item A battery $\mathcal{T}_a$
		%\item Charging and discharching rate $R^-_a, R^+_a$ 
		\item Set of assigned requests $\mathcal{R}_a$ 
	\end{itemize}
\end{frame}




\begin{frame}{Modeling Requests}
	Requests are modeled according to%as tuples $\langle \underline{s'},\bar{t'}, G', P',\lambda, a', b'\rangle$
	\begin{itemize}
		\item  Pick-up and drop-off nodes $\underline{s'} \in \mathcal{V}_n$ and $\bar{t'} \in \mathcal{V}_n$ 
		\item  Transportation demands for goods and people $G'\in \mathbb{R}_{\ge0}$ $P'\in \mathbb{R}_{\ge0}$ 
		\item  Deterministic arrival rate $\lambda \in \mathbb{R}_{>0}$ 
		\item  Time window $[a',b']$ 
	\end{itemize}
\end{frame}



% ==================///==================///==================///

\begin{frame}{The Complete ATS Management Problem}
	%Informally, the Complete ATS Management Problem (CATSM) maximizes number of served requests and minimize vehicle's travel time, while
	Combine the three models to
	\begin{itemize}
		\item Maximize number of served requests
		\item Reduce travel time (and pollution)
	\end{itemize}
	while
	\begin{itemize}
		\item Respecting deadlines
		\item Observing vehicle's characteristics (e.g., charge, cost and capacity)
		\item  Eliminating congestions $\rightarrow$ artificial limit $c$ on vehicles per road.
	\end{itemize}
		\vspace{0.5cm}
	$\rightarrow$ Naturally an optimization problem.
	
\end{frame}
% ==================///==================///==================///

\begin{frame}{Simulating the CATSM in Real-World}
	 \begin{columns}
	 		\begin{column}{0.37\textwidth}
	 			Using real-world data from NYC\\
	 			\begin{itemize}
	 				\item Geo-limited
	 				\item Large road network ($|\mathcal{V}|=$~500, $|\mathcal{E}|=$1700 )
	 				\item Deterministic $\lambda$s %requests rates
	 	 		\end{itemize}
	 		\end{column}
	 			\begin{column}{0.44\textwidth}
	 			\begin{figure}[tbh]
	 				\centering
	 				\begin{tikzpicture}[node distance=0cm]
	 					\node (left) at (0,-1) {\includegraphics[width=0.45\textwidth]{img/new_york_vanilla_info.png}};
	 					\node[right=0.1cm of left, scale=2] {$\Rightarrow$};
	 					\node (right) at (4,-1) {\includegraphics[width=0.45\textwidth]{img/new_york_simplified_roads.png}};
	 				\end{tikzpicture}
	 			\end{figure}
	 			\vspace{0.5cm}
	 		\end{column}
	 \end{columns}
\end{frame}


% ==================///==================///==================///
\begin{frame}{Evaluation}
	\hspace{0.5cm} Promising results, but interrogatives are left open. 
	\vspace{0.5cm}
	\begin{columns}
		\begin{column}{0.35\textwidth}
			
			\begin{itemize}
				\item[+] Solves all the challenges
				\item[+] Reduced computation thanks to a clever rebalancing formulation
				%\item[+] Numerically optimal solutions
				\item[+] Flexible and Modular model
			\end{itemize}
		\end{column}
		%%
		\vline
		\hspace{0.8cm}
		\begin{column}{0.4\textwidth}
			\begin{itemize}
				\item[-] Congestion model is highly simplified
				\item[-] Suffers large networks 
				%\item[-] Not suitable for real-time 
				\item[-] Does not gain insights on the future
			\end{itemize}
		\end{column}
	\end{columns}
\end{frame}