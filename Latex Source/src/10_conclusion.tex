\chapter{Summary and Outlook}

In this work, we only considered one solution to solve the dispatching problem. While the solution found is guaranteed to be optimal, other algorithms could be used as well. For example, one possible approach is to develop an ad-hoc algorithm for it. For instance, requests could be assigned greedily to vehicles according to some heuristics, such as nearest neighbours, or according to the vehicle status, i.e. state of charge, capacity or traveling path. Requests should be assigned in such a way that are still in line with the previous path (see \figref{fig:sens_assignment}). One could, for example, consider as input the previously calculated path for it. \\

In \chapref{ch:mpc}, ride-sharing is not considered. It should be studied its effect on the performance and one direction to persue is to refine the model to account for it. For example, one should also keep track of the vechile capacity as the system progresses, in terms of goods and travelers. \\

You can use GTS also for other purposes. For example, one interesitng idea is to use it to basically automate the design and reconfiguration of vehicles based on the requests. Say you can estimate the requests and what is required from them. For example, if a place is a hospital, you need a vehicle that is able to (i) stop all traffic because it has the priority, (ii) carry people tjat are dyomg and (iii) host the doctors on board as well. In this way, you can basically explore the "design space" of vehicles configurations. At the same time, you can do that with the infrastructure creation / modification. Say you have a roads and possibilities, you want to update traffic lights with the apposite transmission equipment to stop traffic, i.e. V2X. 