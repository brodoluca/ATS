\chapter{Summary and Outlook}\label{ch:summary}
The escalating demands on transportation infrastructure in urban areas necessitate efficient traffic management. Beyond simply facilitating movement, it profoundly impacts economic productivity, environmental sustainability, and societal well-being. Effective traffic control mechanisms are crucial for mitigating the negative effects of an always more dynamic society, by optimizing vehicle flow, reducing congestion, and enhancing economic productivity and air quality. Additionally, they promote social inclusion and improve overall quality of life for urban residents. Motivated by these reasons, this work aims to laying the groundwork for a more efficient and powerful techniques to manage and control transportation systems, which include both people and goods mobility, in the context of smart cities. This work investigates the fundamental methods to achieve this ambitious goal, which are used to cover different aspects of mobility in dense urban environment. Firstly, in \chapref{ch:tim_ats}, a time-invariant model of an ATS is proposed, which takes into consideration the numerous aspects which influence the impact and performance of such a system. Furthermore, in line with the overall scope of the thesis, this model is developed with expansion in mind, namely considering future direction, such as different charging profile for the vehicles and different vehicles category for multi-cargo situations. Furthermore, while in this thesis it was mostly used for urban environment, the model is flexible enough to tackle different domains, such as logistics. This model is then used to reason about the most considered challenges for an ATS, which leads to the formulation of the CATSM problem (\secref{sec:catsm}). Finally, the model and the CATSM are simulated using real-world data to prove their effectiveness and their performance on dense urban environment situations. \\
Motivated by the insights achieved in the previous chapter, specifically regarding the real-time controlling performance, a novel approach for controlling ATS is proposed in \chapref{ch:mpc}. While the use of MPC for controlling ATS has already been investigated in literature, the novelty of this chapter consists of a new linear discrete-time model, which gives the ability to control the speed and track the position of the vehicles in real time. This model is then used to formulate an MPC, which is proved to be stable using an ad-hoc terminal state set and cost function. Furthermore, to further improve real-time performance, a completely new approach is proposed in \secref{sec:gts_mpc} to dynamically reduce the complexity of the road network based on the drop-off and pick-up locations of the requests. Namely, by leveraging graph transformations systems, the RCS, acronym for reduced connectivity schema, is constructed, which is a subset of the original road network. This condensed representation can be utilized to enhance the efficiency of traffic control, improving both overall outcomes and computational efficiency.\\
Although multiple contributions have been made and the most fundamental basis have been established, being this its main goal, this thesis is to consider as the first milestone towards a complete ATS managing systems and multiple research directions are generated as a result. Firstly, while the solution considered in this work to solve the dispatching problem is naturally integrated in further problems' formulation, this highly impacts the computational performance of the solver. This motivates further investigations into other algorithms. An example approach involves devising an ad-hoc algorithm. Requests could be assigned to vehicles greedily based on various heuristics, such as proximity to the nearest neighbor or considering factors like vehicle status (e.g., state of charge, capacity) and current traveling path. It's important that assigned requests align with the vehicle's previous path (refer to \figref{fig:sens_assignment} for a visualization). Previously calculated paths can serve as input for this assignment process, if a step-wise solution for the main challenges is considered. On the same note, a similar consideration for rebalancing vehicles can be done. Furthermore, the introduction of the RCS to this approach would greatly improve the overall performance. Furthermore, the introduction of GTS to the vehicle-centric model could lead to completely new research directions. For example, the model could be revisited to be used, according to ad-hoc defined GTS rules, to prove liveness and safety properties, such as the absence of incidents or guarantee that all requests will be served. Furthermore, GTS could also be used to derive precise requirements regarding elements of the infrastructures, such as the capacity of the road or the charging equipment. Infrastructure elements are also to be considered in this context, since they could be controlled by similar techniques in order to further improve the overall viability of the city.\\
\chapref{ch:mpc} is also source of potential research horizons. Firstly, the model proposed can be further expand to consider ride-sharing. This could be achieved by keeping track of the vehicle capacities as the system progress and, subsequentially, obtain an optimal control law on how to distribute passengers and goods among the vehicles. Its effect should be studied, both in terms of stability and performance. Furthermore, in a smart city scenario, with V2X communication and smart infrastructure, the model can be further expanded to manage multiple factors which have not been considered so far. For example, traffic lights' routine could be defined in real-time based, based on the system's current state, in order to reduce traffic and further optimize road usage. While this is arguably not necessary in a condition of fully autonomous vehicles navigating through the streets, this becomes a beneficial addition in condition of mixed traffic. This scenario also leads to consider the behaviour of non-vehicles circulating through the streets to increase their safety. \\
This potential capillarity is also increased if more powerful models are considered as well, which go beyond linearity, and that consider other aspects of the vehicle, such as its dynamics. This could be used to make optimal local decisions, such as when to cross an intersection, based on the behaviour of other vehicles and their current state. On a macroscopic level, the same method can also be used to take into account other transportation networks, such as railway, to ensure no deadlines are missed by safetly controlling the speed of the vehicles accordingly. These latter aspectes could be considered by applying GTS differently than what proposed in this work. More precisely, new rules could be defined such that the model is dynamically expanded to consider them as well. This discussion outlies the main benefit of the solution proposed in \chapref{ch:mpc} lies in the fact that it allows for a more intricate integration within a smart city, when compared to other methods, since it can account for aspects which are outside of the ATS scope.\\
 Furthermore, the application of GTS also offers multiple directions. An intriguing concept involves automating the design and adaptation of vehicles based on anticipated requests. By estimating the specific requirements for different locations, such as hospitals, vehicles can be configured accordingly. For instance, a vehicle designated for hospital services might need capabilities to \textit{(i)} halt traffic due to priority, \textit{(ii)} accommodate critically ill patients, and \textit{(iii)} facilitate onboard medical staff. This approach allows for exploration of the "design space" for vehicle configurations. Similarly, infrastructure can be modified or created to align with these needs. For example, updating traffic lights with appropriate transmission equipment like V2X can facilitate efficient traffic control. \\
In summary, the insights gained from our discussion, along with the proposed solutions, represent a significant initial step in advancing modern urban landscapes towards safer and more efficient traffic, viability, and transportation systems. Although the challenges ahead are considerable, this thesis establishes sturdy groundwork for future advancements. By utilizing these foundations, we can work towards tangible enhancements in the quality of life for urban residents. This research sets the platform for ongoing innovation and enhancement, providing promising paths for further exploration and implementation. As we move forward, it's crucial to stay dedicated to the overarching objective of developing cities that are not only technologically sophisticated but also supportive of the well-being and prosperity of their inhabitants.