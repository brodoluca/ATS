\chapter{Literature Review}\label{ch:related_work}
\citepaperofs{8883370}{Fehn} modeled a fleet according to energy prices. \citepaperofs{MONTOYA201787}{Montoya} approximated nonlinear charging function using linear piece-wise approximations. According to their result, the error was around 1\%. Furthermore, they introduced the idea of different type of charging stations. \citepaperofs{froger2022electric}{Froger} also made use of a piece-wise linear function for their charging regime. Similarly, also \citepaperofs{kancharla2020electric}{Kancharla}. Other approaches, such as \citepaperofs{9945248}{Nie} do not consider, in their model. the possibilites for vehicles to be recharged.\\
Regarding modeling the battery charging profile, multiple approaches have been proposed. For example, \citepaperofs{electronics9081277}{Han} develop a model based on the internal resistance and voltage of the battery. This model has not been considered in favour of other models, which modeled the battery charging profile according to more fitting factors. \citepaperofs{8115230}{Yu} proposed a more sophisticated model for the charging constraint in general. Their approach, however, differs from the one proposed in this work as they develop it in order to include charging the vehicles during routing, which is discarded in this work in favour of a different approach. \citepaperofs{lee2020model}{Lee} considers a simplified model for the battery which provides states of charge (SOC) in function of the charging current. The author claims his approach to be general enough to work for every charging profile, as long as SoC(t) is a concave and non-decreasing function and SoC(0) = 0. Moreover, there exists an inverse function $\text{SoC}^{-1}(a)$ for $0 < a < Q$.\\

For the vehicle dispatching problem, multiple solutions have been already proposed. For example, \citepaperofs{VASCO2016118}{Vasco} model it as a linear minimum cost multicommodity flow problem; \citepaperofs{holzi}{Holzapfel} model it as a MIP problem. Others, on the other hand, make use of other heuristics. For example, \citepaperofs{LEVIN2017373}{Levin} and \citepaperofs{nn_mora}{Mora} use nearest neighbors. Recently, to improve real-time capabilites, some approaches have been proposed which make use of deep learning, like for e.g. the one proposed by \citepaperofs{8693516}{Yu}.\\
On a basic level, the routing problem can be reconducted to the Vehicle Routing Problem (VRP) (a thorough analysis and explanation can be found in \cite{doi:10.1137/1.9780898718515}). Usually, the VRP is solved and analyzed as a static problem, which implies that requests are known in advance. As a result, origins and destination of each trip is also known a priori. As pointed out by \citepaperofs{amod_review}{Frazzoli}, in AToD systems requests are dynamic, meaning they are not known in advance. More specifically, the task of managing an AToD system can be viewed as a specific case of the dynamic one-to-one pickup and delivery problem. As pointed out by \citepaperof{zhang2016}{Zhang},to provide a more detailed characterization of these systems, several additional attributes and constraints must be considered:
\begin{itemize}
	\item Immediate Service Requests: Requests are made for immediate service rather than being scheduled in advance with specified time windows.
	\item Stochastic Customer Arrivals and Travel Times: The timing of customer requests and the duration of travel are subject to stochastic variability.
	\item Multi-Occupancy Vehicles: Contrary to \cite{zhang2016}, vehicles must be considered to have a capacity, rather than being single-occupancy. This is motivated by the fact that recently developed AV are equipped with more than one seat (see for e.g. \cite{dlr-nemo-bili})
\end{itemize}

\citepaperofs{9294258}{Wollenstein-Betech} propose a solution to the routing-rebalancing problem in mixed traffic situations. They propose an interesting two-graph solutions, i.e. one graph for the AToD system and one for the pedestrians which are interconnected. Furthermore, they also consider private and public vehicles. \\
\citepaperofs{pavone2012robotic}{Pavone} also propose a solution for the rebalancing problem. Although they mainly consider MoD systems, their approach can be easily applied to AToD systems. Their solution, moreover, focuses only on the rebalancing issue, which is studied using a steady-state fluid model, as pointed out in \cite{9294258}. Furthermore, this work does not take into consideration congestions in their model. 


