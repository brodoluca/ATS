\chapter*{Abstract}
Efficient traffic management is critical for modern urban living, impacting economic productivity, environmental sustainability, and societal well-being. With the rise of urbanization and vehicle ownership, traffic inefficiencies have become more pronounced, necessitating innovative solutions. The introduction of autonomous vehicles presents unprecedented opportunities to revolutionize traffic management, leading into the era of Autonomous Transportation Systems (ATS). The thesis proposes a comprehensive framework for ATSs to revolutionize traffic management in urban areas. First, the thesis proposes a newly holistic time-invariant, vehicle-centric model for ATSs, which establishes a flexible and adaptable basis which, already in its basic form, is able to fully describe an ATS and is powerful enough to solve the key challenges of any ATS. Stemming from the insights obtained from this chapter, the subsequent chapter proposes a novel linear time-discrete model for ATSs which, coupled with a model predictive control formulation, proven to be Lyapunov stable, aims at optimize real-world applications of ATSs. Furthermore, to increase the performance of an applied system, a novel strategy based upon graph grammar to drastically reduce the dimension of the model is proposed. Overall, the thesis lays the groundwork for a systematic approach to analyze, control, and optimize ATSs, contributing to the advancement of transportation systems in urban environments.
%Efficient traffic management is critical for modern urban living, impacting economic productivity, environmental sustainability, and societal well-being. With the rise of urbanization and vehicle ownership, traffic inefficiencies have become more pronounced, necessitating innovative solutions. The introduction of Autonomous Vehicles (AVs) presents unprecedented opportunities to revolutionize traffic management, leading into the era of Autonomous Transportation Systems (ATS). The thesis proposes a comprehensive framework for ATSs, built upon four fundamental pillars: optimizing quality-of-service, minimizing road utilization and congestion, addressing operational constraints, and reducing environmental impact. To achieve these objectives, the thesis explores innovative approaches to traffic control, drawing upon advanced technologies and data analytics. First, the thesis proposes a newly holistic time-invariant, vehicle-centric model for ATSs, which builds upon what already established from its predecessors and establishes a flexible and adaptable basis which, already in its basic form, is able to fully describe an ATS. Within this model, the same chapter formulates the Complete ATS Management problem, a completely new formulation which aims at solving all of the key challenges of any ATS. The performance of a simulated ATS is analyzed in the same chapter using real-world data. From the results of this analysis and the insights obtained during the establishment of the model, the following chapter developes a new strategy to manage ATS. More specifically, this chapter proposes novel linear time-discrete model for ATSs which is then used to formulate a Model Predictive Control (MPC), which proven to be stable in the Lyapunov sense, with the objective of further optimize real-world applications of ATS. Furthermore, to increase the performance of an applied system, a novel strategy to drastically reduce the complexity of the model is proposed. This strategy is based upon graph grammar rules and is able to reduce the model dimension to as little as 18 \% of its original size. The thesis concludes with a discussion on future research directions, emphasizing the ongoing exploration of novel techniques to further enhance ATSs' efficiency, sustainability, and societal impact. Overall, the thesis lays the groundwork for a systematic approach to analyze, control, and optimize ATSs, contributing to the advancement of transportation systems in urban environments.
