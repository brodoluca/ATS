\chapter{Motivation}

Traffic management is a cornerstone of modern urban living, intricately woven into the fabric of our daily routines. As populations continue to surge and urbanization accelerates, the demands on transportation infrastructure intensify correspondingly. In the midst of these changes, the need for efficient traffic control and road network management becomes not just desirable, but necessary. At its core, efficient traffic control isn't merely about facilitating the movement of vehicles from point A to point B. On the contrary, it impacts many more aspects of societal well-being economic vitality and environmental sustainability. Consider, for instance, the consequences of traffic inefficiencies, specifically congestions. In fact, not only they waste valuable time, but also incurs substantial costs to individuals and businesses alike. Productivity takes a hit as workers are stopped in gridlocked lanes, deliveries are delayed, and commerce grinds to a sluggish pace. Congestion in the United States exacts a hefty toll on the economy, amounting to approximately 121 billion dollars annually, equivalent to 1\% of the nation's GDP (\cite{schrank2012}). This figure encompasses not just the significant loss of 5.5 billion hours to traffic jams but also the burning of an extra 2.9 billion gallons of fuel. Furthermore, these calculations do not fully consider the potential costs stemming from negative side-effects such as vehicular emissions (including greenhouse gases and particulate matter) (\cite{pant2013}), travel-time uncertainty (\cite{carrion2012}), and an elevated risk of accidents (\cite{hennessy1999}). The exponential growth in urbanization and vehicle ownership has intensified this problem, leading to increased travel times, fuel consumption, air pollution, and stress levels for commuters. Efficient traffic control mechanisms are essential to mitigate these negative consequences and create sustainable, livable urban environments. By optimizing the flow of vehicles, reducing congestion, and minimizing travel times, effective traffic management enhances economic productivity, improves air quality, and fosters healthier, more vibrant communities. Moreover, it promotes equitable access to transportation resources, facilitating social inclusion and enhancing overall quality of life for urban residents. \\
As we move towards the era of smart cities, the integration of Vehicle-to-Everything (V2X) communication systems and the developments in AVs technology offer unprecedented opportunities to revolutionize traffic management. AVs, equipped with advanced sensors and algorithms, have the potential to further enhance traffic control by optimizing routes, adjusting speeds, and minimizing unnecessary stops. These vehicles can navigate through urban environments with precision, adapting to changing road conditions and traffic patterns in ways that human drivers cannot. We are, in other words, moving towards the era of Autonomous Transportation Systems (ATS), where AVs are set to revolutionize urban mobility and deliveries. ATSs must be build upon four main pillars. ATSs must (\textit{i}) ensure optimal quality-of-service by incorporating metrics such as efficiency, reliability and safety; (\textit{ii}) reduce and optimize road utilization while minimzing the number of necessary non-pedestrian zones and congestions, through strategic traffic management and routing algorithms; \textit{(iii)} consider operational constraints, such as charging limitations and vehicle payloads; \textit{(iv)} minimize the environmental impact. 
However, to fully realize the benefits of ATS, it is necessary to lay stable foundations upon which a new way of understanding mobiliy is buit. These fundations include investing in robust infrastructure capable of supporting these technologies, establishing regulatory frameworks to ensure safety and interoperability, and, more importantly, introduce novel models and techniques to manage and control fleet of autonomous vehicles. In light of these considerations, this thesis seeks to explore innovative approaches to traffic control that leverage advanced technologies and data analytics to optimize transportation systems and to ensure that the main pillars of ATSs are respected. Respecting these pillars serves as the theoretical framework for designing and implementing effective ATSs, contributing to the advancement of transportation systems with a focus on efficiency, sustainability, and overall societal well-being.

\section{Thesis Contributions and Organization}
The main objective of this thesis is to establish the necessary foundations towards a systematic approach to analyse, control and optimize ATSs systems with a particular focus on urban environment. The following is a summary of the organization and the contributions in this work. \\
\textbf{\chapref{ch:preliminaries}: } This chapter presents the theoretical background knowledge necessary, focusing on Model Predictive Control and Graph Grammar. \\
\textbf{\chapref{ch:tim_ats}: } An holistic time-invariant, vehicle centric model for ATSs is proposed. This is a general, yet efficient and adaptable, representation of an ATS. Furthermore, the Complete ATS Management problem is formulated, which, within the proposed model, aims at solving the main challenges in ATSs. \\
\textbf{\chapref{ch:mpc}: } This chapter develops a novel linear time-discrete model for ATS with the main objective of an optimized real-time application. In this regard, an MPC formulation is derived and proven to be stable in the Lyapunov sense. Furthermore, a possible optimization with GTS is proposed. \\
\textbf{\chapref{ch:summary}: } Multiple directions for future research are still to be explored. This chapter includes a discussion and future outlooks. 